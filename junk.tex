Deep neural networks as one of the machine learning categories have been proposed
since 1980s and proved successful in recent years. This thesis proposes methods which
leverage deep neural networks to solve the camera geo-calibration problem, which is
one of the import computer vision tasks. Furthermore, we also use the geo-calibration
information (geo-location, orientation, etc.) to boost the training of the deep neural
networks for image understanding.
Our work focuses on three primary research areas: 1) estimating vanishing points
and horizon line using image global context computed with deep neural networks. 2)
domain adaptation from the ground image semantics to the satellite image semantics,
and 3) learning transient feature representations for webcam images using the time
and geo-location information in a semi-supervised manner.
The primary contribution of this thesis is to explore the relation between the
camera calibration information and image semantics, and bridge the gap between
two research domains using deep neural networks, which provides us new ways for
camera calibration and network training.


abstract
Camera geometry is one of the fundamental elements in computer vision.
It plays essential roles in many applications like depth estimation
and image mensuration.
%
In recent years, the rise of deep neural networks (DNNs) had changed
the game in many computer vision areas, their ability of abstracting
high-level semantic knowledges makes a lot of hard challenges
solvable.
%
This thesis mainly explores the possibility of combining camera
geometry and DNNs. That is 1) Detecting camera geometry using DNNs,
and 2) Training DNNs with camera geometric information.  Our work
includes two research targets: The first part of this thesis focuses
on camera geometry detection with DNNs, and the rest of it
demonstrates how camera geometry can boost the learning of DNNs.
%
The primary contribution of this thesis is to explore the relation
between camera geometry and image semantics, and bridge the gap
between these two research domains, which provides us new ways for
both camera calibration and deep neural network training.
