\chapter{Introduction}
\label{chap:intro}

\makeatletter
\newenvironment{chapquote}[2][2em]
{\setlength{\@tempdima}{#1} \def\chapquote@author{#2} \parshape 1
  \@tempdima \dimexpr\textwidth-2\@tempdima\relax \itshape}
{\par\normalfont\hfill--\
\chapquote@author\hspace*{\@tempdima}\par\bigskip}
\makeatother

\begin{chapquote}{Alan Turing}
  ``I propose to consider the question: Can machine think?''
\end{chapquote}

Before the father of computer asked this question, creating
machines that can think as human has been a dream through the human
history. In the late 18th century, Hungarian inventor, Wolfgan von
Kempelen, constructed an automatic machine called ``The Turk'' (also
known as the ``Mechanical Turk'') to play a chess game. Though it
proved to be an elaborate hoax, people's enthusiasm for machine
intelligence lasts till today. As one of the core elements of
intelligence, human vision plays an important role in our daily lives.
In order to make computer understand images, scholars had put a lot of
research efforts in the study of computer vision. In 1960s, MIT
professor Dr. Marvin Minsky asked his undergraduates to develop
programs that can recognize objects from a scene in their summer
project~\cite{boden2006mind}. Half a century has passed, the question
of this ``summer project'' still remains open.

The ultimate goal of image understanding is to transfer the visual
images into numerical or symbolic descriptions of the world
that are helpful for other thought processes for decision making.
In general, understanding images is not trivial for machines.
For different application scenarios, researchers divide image
understanding into different computer vision tasks. For example,
pedestrian detection is essential to autonomous driving, and image
retrieval to an image searching engine.

Among algorithms that are developed to address these sub-tasks, deep
neural networks (DNNs) rise as stars in recent years. Methods that
uses DNNs achieve state-of-the-art performances in many computer
vision areas (\todo{citations}). As parametric algorithms, deep neural
networks require training for their parameters. Methods to train DNNs
include supervised algorithms, unsupervised algorithms, and
reinforcement learning. Supervised algorithms are the most popular and
commonly applied in many computer vision areas, thanks to their good
performances and the fact that they are easy to implement. However,
supervised learning approaches are sensitive to the quality of data
annotations. In some applications like overhead image segmentation and
transient attributes estimation, data with noise-free annotation is
either limited in quantity or technically difficult to acquire.
In this thesis, Our methods focus on lavaging the camera geometric
information to deal with the challenges caused by the lack of
noise-free annotations during DNN training. The insight behind it is
that the camera geometry affects the way that images manifest
objects/scenes, so it is possible to exploit the relation between the
camera geometry and image semantics to understand images better.

In computer vision, the geometric model of camera projection 
 
In the camera pin-hole model, camera geometric parameters consists of
the camera intrinsics and the extrinsics. The intrinsic parameters of
a camera encode information of the camera itself such as focal length,
principal point \etc; the extrinsic parameters are referred to as the
external states of the camera, such as the camera poses and
offset/location under some coordinate system. In this work, our study
of the camera geometry mainly includes the extrinsic like camera pose
and the geo-orientation/geo-localization of the camera.

Many research has been done to detect the camera
geometry from input images. \todo{citation and examples}. Most of the
previous work optimize for a limited set of camera calibration
parameters only. However, our research found that there exists an
effective way of optimizing multiple camera calibration parameters at
simultaneously (refer to \chapref{mcmc}). Our algorithm uses the Monte
Carlo Markov Chain (MCMC) method to sample the possible solutions that
minimize the objective function, which measures the ``matchness''
between the observation (input image) and the camera calibration.

Among the most influential approaches for camera geometry estimation,
most methods are based on traditional statistics diagram. Not until a
fewer years ago, machine learning based methods gradually caught
people's attention in this area, thanks for the popularity of deep
neural network. Neural networks were first proposed in 80th
(\todo{citation}), The back-forward method developed by Hinton
\todo{citation} makes the effective training possible. However, neural
networks are generally thought by researchers merely toy that can only
solve a very limited set of tasks. It is the develop of computational
resources and the big collection of annotated data that enables the
neural networks to grow deeper and eventually brings it to the stage of
industrial applications.

In computer vision, low-level elements like edges, line segments,
super-pixels and vanishing points play important roles in many
applications like image measuration, facade detection, and
geo-localization.
Though the CNNs are good at extracting high-level information from
images, more and more researches start focusing on using CNN to
extract low-level these elements.
%
Lee \etal~\cite{lee2017semantic} propose to detect semantic line
segments, which are able to segment the scene semantically, using
neural network. Maninis \etal~\cite{maninis2016convolutional} and Yang
\etal~\cite{yang2016object} explore to detect object contours with
CNNs.
%
Aside from edges and contours, another category of low-level
information that the CNNs can detect is the camera geometry, like
camera location, vanishing points, and horizon lines.
%
In the area of vanishing point detection, we are among the first
researches that use deep neural networks to solve the problem. The
deep neural networks prove to be a good tool to extract high-level
information that offer great help to conquer the problems in camera
geometry estimation. \todo{many citations}.

In our previous work, we have shown that deep neural networks are
powerful tools for camera geometry estimation as well. In fact, the
camera geometry can also assist learning high-level image semantics.
Hoiem \etal~\cite{hoiem2008putting} demonstrate that taking camera
perspective (horizon line) into consideration improves the performance
for object detection. Godard \etal~\cite{godard2017unsupervised} learn
to predict the scene depth from a single image using left-right
consistency, which is also an application of camera geometry.
\todo{more citations about the projective loss}.

Another example of camera geometry boosting the learning project is to
exploit the overhead imagery. Using the geometric locations of the
ground-view images we can download the corresponding overhead images
at the same locations through public online resources like bingmap
\todo{cite}. We refer to these image pairs as ``cross-view image
pairs''.

\todo{copy from scott} Efforts have been made to
automate overhead image analysis. As early as
1970~\cite{idelsohn1970learning} methods were introduced for
classifying terrain types from a single overhead image, with the goal
of automatically generating terrain maps.  Similarly, in 1976 Bajcsy
et al.~\cite{bajcsy1976computer} described a system for recognizing
roads, intersections, and other road-like objects in overhead imagery.
However, as overhead imaging differs drastically from ground-level
imaging, the majority of techniques that have been developed have
occurred independently and in task-specific ways~\cite{Rozen}.
%\brief{overhead image grows fast}.

Our work demonstrates that we are able to transfer the learned semantic
knowledge from the ground image domain to the overhead image domain,
given the camera geometric information of these two kind of images. In
our algorithm, the camera geometry helps to collect the ``cross-view
image pairs'' and aligned the orientation directions of images in a
pair.

Camera geometry can also be used as weak annotations during training
of the networks for image understanding.  One of the key to the
success of machine learning approaches are the large quantity of
annotated data. However, fully annotated imagery consists of only a
tiny part of the image resources available online. The idea is to
exploit other meta-info that are automatically recorded/captured by
devices to train the ML models. One of the meta-info potentially
helpful for learning is camera geometry.  We demonstrate that the
camera geometry can be used as weak ground truth for supervision
problem. By learning the capture time and the geolocation of the
images simultaneously, our network is able capture the geo-temporal
related high-level features.


\section{Our Approach}
My thesis focuses on the image understanding with camera geometry.
There are two directions we will address: 1) How to estimate camera
geometry from images; 2) How can we use camera geometry to improve the
current methods for image semantic extracting. For the first direction,
we explore both the traditional method and deep learning approaches on
different geometric calibration parameters. For the second direction,
we will exploit the image geometric location to help extract
semantically meaningful informations from images.

\begin{itemize}[noitemsep]

  \item \textbf{Camera Geometry Detection:} 
  We investigate different methods to detect camera geometry from the
  input images. We first explore a sampling approach to find the
  complete camera calibration parameters that fit the observation of
  objects (input image) best. In our next work, we use a deep neural
  network to learn to detect horizon line and vanishing points. These
  two approaches provides information for the next stage of our
  research.

  \item \textbf{Camera Geometry for Image Understanding:}
  We explore the potentials of camera geometry for image understanding.
  Current machine learning methods address problems that involves mostly
  the ground imagery. 
  Studies around overhead images are relatively less thus much fewer
  annotated data exists in public. With the camera geometric
  information, we can pair the ground imagery with overhead imagery to
  form cross-view pairs. Our algorithm explore how to transfer the
  learned knowledge from a network designed for ground level image
  semantic segmentation to a new network designed for overhead imagery
  semantic segmentation use these cross-view pairs.
  Furthermore, we also explore the potential of the camera geometry as
  weak annotation in supervised learning. Our algorithm shows that by
  directly learning to predict camera locations, the network networks
  are able to extract high-level geo-temporal features.

\end{itemize}


\section{Synopsis}

The remainder of this work is organized as follows:
  
\begin{itemize}[noitemsep]

  \item \textbf{\chapref{mcmc} - 
  Complete Camera Geo-Calibration with MCMC Method:} \newline \newline
  The complete camera geo-calibration includes the detection of camera
  location in the world, the pose and the focal length of the camera
  given the input image. By constructing a objective function that
  measures the ``matchness'' between the camera calibration and the
  observation of the image, we are able to optimize the parameters in
  the calibration space. However, this space multi-dimensional, which
  makes the naive grid search infeasible. Our insight is to do the
  optimization using Markov Chain Monte Carlo (MCMC) method. Our
  algorithm is able to avoid unnecessary sampling in the
  low-probabilistic region in the parameter space thus reduce the
  computational time dramatically. \newline

  \item \textbf{\chapref{fasthorizon} -
  Detecting Horizon and VPs using CNN:} \newline \newline
  Horizon line and vanishing points (VPs) plays important roles in many
  application such as image mensuration and facade detection. In the
  previous work, researchers tended to use build bottom-up approaches to
  solve this problem. Our work is inspired by human's perception of the
  horizon line and the vanishing point. Our method exploit the image
  global context to provide a prior distribution of the final solution,
  thus to reduce the dimensionality of the solution space. \newline

  \item \textbf{\chapref{crosstransf} -
  Cross-view Domain Adaptation:} \newline \newline
  Current machine learning methods address problems that involves mostly
  the ground imagery. Tons of dataset targeting on ground imagery are
  annotated while much fewer are done for the overhead imagery.
  In this work, we try to transfer the semantics from the ground imagery
  to the overhead imagery by identifying the latent geometry
  correspondences between these two. Similar to methods that driven by
  the projective losses \todo{citation}, our network learns 
  both the geometric projection between the overhead and ground
  images, and the semantics of the overhead images jointly. \newline

  \item \textbf{\chapref{whenwhere} -
  Extract High-Level Feature by Learning When and Where:} \newline \newline
  One of the key to the success of machine learning approaches are the
  large quantity of annotated data. However, fully annotated imagery
  only consists of a tiny part of the image resources available online.
  The idea is to exploit other meta-info that are automatically
  recorded/captured by devices to train the ML models.  We demonstrate
  that the camera geometry can be used as weak ground truth for
  supervision problem. By learning the capture time and the geolocation
  of the images, our network is able capture the geo-temporal related
  high-level features. \newline

\end{itemize}
