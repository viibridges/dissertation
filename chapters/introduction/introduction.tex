\chapter{Introduction}
\label{chap:intro}

\makeatletter
\newenvironment{chapquote}[2][2em]
{\setlength{\@tempdima}{#1} \def\chapquote@author{#2} \parshape 1
  \@tempdima \dimexpr\textwidth-2\@tempdima\relax \itshape}
{\par\normalfont\hfill--\
\chapquote@author\hspace*{\@tempdima}\par\bigskip}
\makeatother

%\begin{chapquote}{Walt Disney}
  %``Of all of our inventions for mass communication, pictures still
  %speak the most universally understood language.''
%\end{chapquote}

%Outline:
\begin{itemize}
  \item sliver bullet and ``summer project'' story
  \item ``summer project'' is still going on
  \item tranditional CV methods and emerge of deep learning (gamer changer)
  \item geometry information are useful for applications, how to learn them
  \item geometry information are useful for image understanding (training), how to use them
  \item overhead image (encoding location) as type of geometry information are useful
  \item growing, densely available in space and time
\end{itemize}

\brief{
The topic of understanding images has been around for half a century.
A well known [annodose] in computer vision community is set in the
60s. The MIT professor Dr. Marvin Minsky assigned a couple of
undergrads to spend the summer programming a computer to recognize
objects in the scene~\cite{boden2006mind}. Quite like the attempts of
proving the famous ``Four color theorem'', [[story about a prof claim
he can prove it in a class]]. Fortunate for the mathematicans, the
``Four color theorem'' has been proved by using powerful computational
resources, while the computer scientists are still trapped in Dr.
Marvin Minsky's summer program (not fully solved yet).
}

\brief{
The ultimate goal of image understanding is to extract high-level
understanding from digital images or videos. The task seems natural
and easy for an adult human, but it is difficult for machines. 1) Silver
bullet does not exist \todo{cite}, no general algorithm, we do things
in a ad-hoc manner.
}

\brief{
The emerge of deep neural network brings light to the dream. The
neural network is a biological design, just like our brains. [history
of neural network]. Data and computational power bring us here.
}

In computer vision, low-level elements like edges, line segments,
super-pixels and vanishing points play important roles in many
applications like image measuration, facade detection, and
geo-localization.
Though the CNNs are good at extracting high-level information from
images, more and more researches start focusing on using CNN to
extract low-level these elements.
%
Lee \etal~\cite{lee2017semantic} propose to detect semantic line
segments, which are able to segment the scene semantically, using
neural network. Maninis \etal~\cite{maninis2016convolutional} and Yang
\etal~\cite{yang2016object} explore to detect object contours with
CNNs.
%
Aside from edges and contours, another category of low-level
information that the CNNs can detect is the camera geometry, like
camera location, vanishing points, and horizon lines.
Zhai \etal~\cite{zhai2016horizon} Firstly try to detect horizon
line and vanishing point using global image context extracted by CNN.
Workman \etal~\cite{workman2016horizon} take the idea further and
design an end-to-end network to predict horizon line from images
in the wild. 
Hold-Geoffroy \etal~\cite{hold2017perceptual}
improve~\cite{workman2016horizon}'s results by using deeper networks,
and introduce a new perceptual measure for horizon line detection.
Unlike the previous work that mainly focus on horizon line detection,
Zhang \etal~\cite{zhang2018dominant} propose to detect the dominant
vanishing point using deep neural networks.

\brief{
Geometric information can help training neural network to extract
high-level information.
}
Camera geometry can provide help learn high-level image semantics.
Hoiem \etal~\cite{hoiem2008putting} demonstrate that taking camera
perspective (horizon line) into consideration improves the performance
for object detection.

\todo{re-edit}
The use of overhead imagery has been proved useful for image
understanding.
Several methods have been recently proposed to jointly reason about
co-located aerial and ground image pairs. Luo
\etal~\cite{luo2008event} demonstrate that aerial imagery can aid
in recognizing the visual content of a geo-tagged ground image.
M{\'a}ttyus \etal~\cite{mattyus2016hd} perform joint inference over
monocular aerial imagery and stereo ground images for fine-grained
road segmentation. Wegner \etal~\cite{wegner2016cataloging} build a
map of street trees. Given the horizon line and the camera intrinsics,
Ghouaiel and Lef{\`e}vre~\cite{ghouaiel2016coupling} transform
geo-tagged ground-level panoramas to a top-down view to enable
comparisons with aerial imagery for the task of change detection.
Recent work on cross-view image
geolocalization~\cite{lin2013cross,lin2015learning,workman2015geocnn,workman2015wide}
 has shown that convolutional neural
networks are capable of extracting features from aerial imagery
that can be matched to features extracted from ground imagery.
Vo \etal~\cite{vo2016localizing} extend this line of work,
demonstrating improved geolocalization performance by applying an
auxiliary loss function to regress the ground-level camera orientation
with respect to the aerial image. To our knowledge, our work is the
first work to explore predicting the semantic layout of a ground
image from an aerial image.

\brief{
We use geometric information and DL to solve computer vision problems.
}


\section{Our Approach}
My thesis focuses on the image understanding with camera geometry.
There are two directions we will address: 1) How can we obtain camera
geometry from images; 2) How can we use camera geometry to improve the
current methods for image semantic extracting. For the first direction,
we explore both the traditional method and deep learning approaches on
different geometric calibration parameters. For the second direction,
we will exploit the image geometric location to help extract
semantically meaningful informations from images.

\begin{itemize}[noitemsep]

  \item \textbf{Camera Geometry Detection:} 
  We investigate different methods to detect camera geometry from the
  input images. We first explore a sampling approach to find the
  complete camera calibration parameters that fit the observation of
  objects (input image) best. In our next work, we use a deep neural
  network to learn to detect horizon line and vanishing points. These
  two approaches provides information for the next stage of our
  research.

  \item \textbf{Camera Geometry for Image Understanding:}
  We explore the potentials of camera geometry for image understanding.
  Current machine learning methods address problems that involves mostly
  the ground imagery. Tons of dataset targeting on ground imagery are
  annotated while much fewer are done for the overhead imagery.

\end{itemize}


\section{Synopsis}

The remainder of this work is organized as follows:
  
\begin{itemize}[noitemsep]

  \item \textbf{\chapref{dependence} - Are Deep Image Representations
    Geo-Informative?} \newline \newline In this chapter, we
    investigate the usefulness of deep image representations,
    extracted from convolutional neural networks applied to
    traditional vision tasks, for problems in geospatial image
    analysis. In particular, we analyze their discriminative ability
    with regard to location through several problem settings,
    including region identification in ground-level imagery,
    understanding and interpreting overhead images, and cross-view
    image matching.  Our results demonstrate the effectiveness of deep
    image representations extracted from CNNs, on both ground-level
    and overhead imagery, for capturing geographically discriminative
    features relating image appearance to geographic location.  This
    points to a promising direction for future research in building
    deep-learning based models that are directly targeted at problems
    of localization and location-related feature extraction from
    ground-level and overhead imagery. \newline


  \item \textbf{\chapref{mcmc} - 
  Complete Camera Geo-Calibration with MCMC Method:} \newline \newline
  The complete camera geo-calibration includes the detection of camera
  location in the world, the pose and the focal length of the camera
  given the input image. By constructing a objective function that
  measures the ``matchness'' between the camera calibration and the
  observation of the image, we are able to optimize the parameters in
  the calibration space. However, this space multi-dimensional, which
  makes the naive grid search infeasible. Our insight is to do the
  optimization using Markov Chain Monte Carlo (MCMC) method. Our
  algorithm is able to avoid unnecessary sampling in the
  low-probabilistic region in the parameter space thus reduce the
  computational time dramatically. \newline

  \item \textbf{\chapref{fasthorizon} -
  Detecting Horizon and VPs using CNN:} \newline \newline
  Horizon line and vanishing points (VPs) plays important roles in many
  application such as image mensuration and facade detection. In the
  previous work, researchers tended to use build bottom-up approaches to
  solve this problem. Our work is inspired by human's perception of the
  horizon line and the vanishing point. Our method exploit the image
  global context to provide a prior distribution of the final solution,
  thus to reduce the dimensionality of the solution space. \newline

  \item \textbf{\chapref{crosstransf} -
  Cross-view Domain Adaptation:} \newline \newline
  Current machine learning methods address problems that involves mostly
  the ground imagery. Tons of dataset targeting on ground imagery are
  annotated while much fewer are done for the overhead imagery.
  In this work, we try to transfer the semantics from the ground imagery
  to the overhead imagery by identifying the latent geometry
  correspondences between these two. Similar to methods that driven by
  the projection losses \todo{citation}, our network learns 
  both the geometric projection between the overhead and ground
  images, and the semantics of the overhead images jointly. \newline

  \item \textbf{\chapref{whenwhere} -
  Extract High-Level Feature by Learning When and Where:} \newline \newline
  One of the key to the success of machine learning approaches are the
  large quantity of annotated data. However, fully annotated imagery
  only consists of a tiny part of the image resources available online.
  The idea is to exploit other meta-info that are automatically
  recorded/captured by devices to train the ML models.  We demonstrate
  that the camera geometry can be used as weak ground truth for
  supervision problem. By learning the capture time and the geolocation
  of the images, our network is able capture the geo-temporal related
  high-level features. \newline

\end{itemize}
