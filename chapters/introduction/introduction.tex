\chapter{Introduction}
\label{chap:intro}

\makeatletter
\newenvironment{chapquote}[2][2em]
{\setlength{\@tempdima}{#1} \def\chapquote@author{#2} \parshape 1
  \@tempdima \dimexpr\textwidth-2\@tempdima\relax \itshape}
{\par\normalfont\hfill--\
\chapquote@author\hspace*{\@tempdima}\par\bigskip}
\makeatother

%\begin{chapquote}{Walt Disney}
  %``Of all of our inventions for mass communication, pictures still
  %speak the most universally understood language.''
%\end{chapquote}

%Outline:
\begin{itemize}
  \item sliver bullet and ``summer project'' story
  \item ``summer project'' is still going on
  \item tranditional CV methods and emerge of deep learning (gamer changer)
  \item geometry information are useful for applications, how to learn them
  \item geometry information are useful for image understanding (training), how to use them
  \item overhead image (encoding location) as type of geometry information are useful
  \item growing, densely available in space and time
\end{itemize}

The topic of understanding images has been around for half a century.
A well known [annodose] in computer vision community is set in the
60s. The MIT professor Dr. Marvin Minsky assigned a couple of
undergrads to spend the summer programming a computer to recognize
objects in the scene~\cite{boden2006mind}. Quite like the attempts of
proving the famous ``Four color theorem'', [[story about a prof claim
he can prove it in a class]]. Fortunate for the mathematicans, the
``Four color theorem'' has been proved by using powerful computational
resources, while the computer scientists are still trapped in Dr.
Marvin Minsky's summer program (not fully solved yet).

The ultimate goal of image understanding is to extract high-level
understanding from digital images or videos. The task seems natural
and easy for an adult human, but it is difficult for machines. 1) Silver
bullet does not exist \todo{cite}, no general algorithm, we do things
in a ad-hoc manner.

The emerge of deep neural network brings light to the dream. The
neural network is a biological design, just like our brains. [history
of neural network]. Data and computational power bring us here.

Lots algorithm focus on extract high-level info from the images. More
and more deep learning algorithm start identifying lower-level info
from the image like edge extracting and vanishing point detection.
(geometric info). Identifying geometric information is useful for many
applications.

Geometric information can help training neural network to extract
high-level information.

Specifically, Use of overhead image (or map layout) is useful.
overhead imagery are useful and easy to access, gowing densely
available in space and time.

We use geometric information and DL to solve computer vision problems.


\section{Our Approach}
Our approach conver the following area:

\begin{itemize}[noitemsep]

  \item \textbf{Extract Geometric Info from Image using DL}

  \item \textbf{Overhead Image for Image Understanding}

  \item \textbf{Extract High-Level Feature by Learning When and Where.}

\end{itemize}


\section{Synopsis}

The remainder of this work is organized as follows:
  
\begin{itemize}[noitemsep]

  \item \textbf{\chapref{dependence} - Are Deep Image Representations
    Geo-Informative?} \newline \newline In this chapter, we
    investigate the usefulness of deep image representations,
    extracted from convolutional neural networks applied to
    traditional vision tasks, for problems in geospatial image
    analysis. In particular, we analyze their discriminative ability
    with regard to location through several problem settings,
    including region identification in ground-level imagery,
    understanding and interpreting overhead images, and cross-view
    image matching.  Our results demonstrate the effectiveness of deep
    image representations extracted from CNNs, on both ground-level
    and overhead imagery, for capturing geographically discriminative
    features relating image appearance to geographic location.  This
    points to a promising direction for future research in building
    deep-learning based models that are directly targeted at problems
    of localization and location-related feature extraction from
    ground-level and overhead imagery.  This work was originally
    reported in~\cite{workman2015geocnn}. \newline

\end{itemize}
