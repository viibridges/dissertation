\documentclass[final]{ukthesis}

\usepackage{times}
\usepackage{epsfig}
\usepackage{graphicx}
\usepackage{paralist}
\usepackage{amsmath}
\usepackage{amssymb}
\usepackage{caption}
\usepackage{subcaption}
\usepackage{import}
\usepackage{xspace}
\usepackage{titlesec}
\usepackage{enumitem}
\usepackage{lipsum}
\usepackage{memhfixc}
\usepackage{algorithm2e, algorithmic}
\usepackage{soul}
\usepackage{multirow}
\usepackage{miscstyle}


\usepackage[pdfauthor={Menghua Zhai},
            pdftitle={dissertation},
            bookmarks=false,
            bookmarksnumbered=true,
            bookmarksopen=true,
            breaklinks=true,
            letterpaper=true,
            linktocpage,
            pagebackref=true]{hyperref}

\begin{document}
\emergencystretch 3em

\author{Menghua Zhai}
\title{Lavaging Camera Geometry in Deep Neural Network Learning for Image Understanding}

\abstract
{ \SingleSpacing
In recent years, Deep Neural Networks (DNNs) have proven to be very
successful in many computer vision areas. Among the most popular
training techniques, supervised learning is wildly used for DNN
training and easy to implement. However, successful supervised
learning approaches are sensitive to the quality of data
annotations. In some applications like overhead image segmentation and
transient attributes estimation, data with noise-free annotation is
either limited in quantity or technically difficult to acquire.
In this thesis, we use domain adaptation and weak supervision to
address this problem. Our methods focus on lavaging the camera
geometric information to deal with the challenges caused by the lack
of noise-free annotations during DNN training.
Since the camera geometry affects the way that images manifest
objects/scenes, it is possible to exploit the relation between the
camera geometry and image semantics to understand images better.
Our approach consists of two parts: 1) we explore new methods
to estimate the camera geometric information from images, and 2) 
we propose novel algorithms that lavages the camera geometry to train
deep neural networks.
The primary contribution of our thesis is to relate the camera
geometry and image semantics through deep neural networks. Our
discoveries can provide researchers new insights for both camera
calibration and DNN training.
}
\keywords{ camera geometry, deep neural networks, computer vision}

\advisor{Nathan Jacobs}
\dgs{Miroslaw Truszczynski}

\frontmatter
\maketitle

%\begin{acknowledgments}
	%Acknowledge people/things here
%\end{acknowledgments}

%\begin{dedication}
%	Dedicated to things (optional)
%\end{dedication}

\tableofcontents\clearpage
%\listoffigures\clearpage
%\listoftables\clearpage

\mainmatter

\inject{introduction}
%\inject{background}

%\inject{mcmc}
%\inject{fasthorizon}
%\inject{crosstransf}
%\inject{whenwhere}

%\inject{discussion}

\backmatter

\bibliographystyle{plain}
\bibliography{biblio}

%\inject{vita}

\end{document}
