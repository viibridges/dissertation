\documentclass[final]{ukthesis}

\usepackage{times}
\usepackage{epsfig}
\usepackage{graphicx}
\usepackage{paralist}
\usepackage{amsmath}
\usepackage{amssymb}
\usepackage{caption}
\usepackage{subcaption}
\usepackage{import}
\usepackage{xspace}
%\usepackage{titlesec}
\usepackage{enumitem}
\usepackage{lipsum}
\usepackage{memhfixc}
\usepackage{algorithm2e, algorithmic}
\usepackage{soul}
\usepackage{multirow}
\usepackage{miscstyle}


\usepackage[pdfauthor={Menghua Zhai},
            pdftitle={dissertation},
            bookmarks=true,
            bookmarksnumbered=true,
            bookmarksopen=true,
            breaklinks=true,
            letterpaper=true,
            linktocpage,
            pagebackref=true]{hyperref}

\begin{document}
\emergencystretch 3em

\author{Menghua Zhai}
\orcid{0000-0001-5874-238X}
\title{Deep Probabilistic Models for Camera Geo-Calibration}

\abstract
{ \SingleSpacing
The ultimate goal of image understanding is to transfer visual
images into numerical or symbolic descriptions of the scene
that are helpful for decision making.
Knowing when, where, and in which direction a picture was taken, the
task of geo-calibration makes it possible to use imagery to understand
the world and how it changes in time. Current models for
geo-calibration are mostly deterministic, which in many cases fails to
model the inherent uncertainties when the image content is
ambiguous. Furthermore, without a proper modeling of the uncertainty, subsequent
processing can yield overly confident predictions.
To address these limitations, we propose a probabilistic model for
camera geo-calibration using deep neural networks.
While our primary contribution is geo-calibration, we also show that
learning to geo-calibrate a camera allows us to implicitly learn to
understand the content of the scene.
}
\keywords{computer vision, geo-calibration, deep neural networks}

\advisor{Nathan Jacobs}
\dgs{Miroslaw Truszczynski}

\frontmatter
\maketitle

\begin{acknowledgments}
  I would like to express the deepest appreciation to my advisor,
Professor Nathan Jacobs, who not only financed my Ph.D. career, but
also supported my research with his profound insight
and knowledge. Over the years I have benefited from guidance,
expertise, and patience. I have also learned a lot from his efficient
work style and the way he attacks problems. I honestly fell lucky to
have had the opportunity to work alongside him and I am proud of what
we have accomplished together.

I would like to take this opportunity to give my thanks to several
individuals who helped me through the path:  Jinze Liu, who
first got me enrolled to the graduate school; 
Qiang Ye, Judy Goldsmith, and Andrew Klapper, for intriguing
my desire for higher-level knowledge. A huge appreciation to 
other members of my advisory committee, Ruigang Yang and Ramakanth
Kavuluru, for their invaluable feedback during my defense.

I have had the privilege of working and collaborating with many
individuals, including: Zachary Bessinger, Tawfiq Salem, Connor
Greenwell, Ryan Baltenberger, Neil Moore, Samuel Schulter, and others.
In the end, I would like to thank my friend and colleague, Scott
Workman, with whom I worked closest and accomplished a great deal of
research projects.
\end{acknowledgments}

%\begin{dedication}
%	Dedicated to things (optional)
%\end{dedication}

\tableofcontents\clearpage
\listoffigures\clearpage
\listoftables\clearpage

\mainmatter

\inject{introduction}

\inject{mcmc}
\inject{fasthorizon}
\inject{whenwhere}
\inject{crosstransf}

\inject{discussion}

\backmatter

\bibliographystyle{plain}
\bibliography{biblio}

\inject{vita}

\end{document}
