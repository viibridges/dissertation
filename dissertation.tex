\documentclass[final]{ukthesis}

\usepackage{times}
\usepackage{epsfig}
\usepackage{graphicx}
\usepackage{paralist}
\usepackage{amsmath}
\usepackage{amssymb}
\usepackage{caption}
\usepackage{subcaption}
\usepackage{import}
\usepackage{xspace}
\usepackage{titlesec}
\usepackage{enumitem}
\usepackage{lipsum}
\usepackage{memhfixc}
\usepackage{algorithm2e, algorithmic}
\usepackage{soul}
\usepackage{multirow}
\usepackage{miscstyle}


\usepackage[pdfauthor={Menghua Zhai},
            pdftitle={dissertation},
            bookmarks=false,
            bookmarksnumbered=true,
            bookmarksopen=true,
            breaklinks=true,
            letterpaper=true,
            linktocpage,
            pagebackref=true]{hyperref}

\begin{document}
\emergencystretch 3em

\author{Menghua Zhai}
%\title{Building a Probabilistic Model with Deep Neural Networks for Camera Geo-Calibration}
%\title{Statistical Models for Camera Geo-Calibration}
\title{Deep Probabilistic Models for Camera Geo-Calibration}

\abstract
{ \SingleSpacing
The ultimate goal of image understanding is to transfer visual
images into numerical or symbolic descriptions of the scene
that are helpful for decision making.
Knowing when, where, and in which direction a picture was taken, the
task of geo-calibration makes it possible to use imagery to understand
the world and how it changes in time. Current models for
geo-calibration are mostly deterministic, which in many cases fails to
model the inherent uncertainties when the image content is
ambiguous. Furthermore, without a proper modeling of the uncertainty, subsequent
processing can yield overly confident predictions.
To address these limitations, we propose a probabilistic model for
camera geo-calibration using deep neural networks.
While our primary contribution is geo-calibration, we also show that
learning to geo-calibrate a camera allows us to implicitly learn to
understand the content of the scene.
}
\keywords{computer vision, geo-calibration, deep neural networks}

\advisor{Nathan Jacobs}
\dgs{Miroslaw Truszczynski}

\frontmatter
\maketitle

%\begin{acknowledgments}
	%Acknowledge people/things here
%\end{acknowledgments}

%\begin{dedication}
%	Dedicated to things (optional)
%\end{dedication}

\tableofcontents\clearpage
%\listoffigures\clearpage
%\listoftables\clearpage

\mainmatter

\inject{introduction}

%\inject{mcmc}
%\inject{fasthorizon}
%\inject{crosstransf}
%\inject{whenwhere}

\inject{discussion}

\backmatter

\bibliographystyle{plain}
\bibliography{biblio}

%\inject{vita}

\end{document}
