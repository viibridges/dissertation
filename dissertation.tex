\documentclass[final]{ukthesis}

\usepackage{times}
\usepackage{epsfig}
\usepackage{graphicx}
\usepackage{paralist}
\usepackage{amsmath}
\usepackage{amssymb}
\usepackage{caption}
\usepackage{subcaption}
\usepackage{import}
\usepackage{xspace}
\usepackage{titlesec}
\usepackage{enumitem}
\usepackage{lipsum}
\usepackage{memhfixc}
\usepackage{algorithm2e, algorithmic}
\usepackage{soul}
\usepackage{multirow}
\usepackage{miscstyle}


\usepackage[pdfauthor={Menghua Zhai},
            pdftitle={dissertation},
            bookmarks=false,
            bookmarksnumbered=true,
            bookmarksopen=true,
            breaklinks=true,
            letterpaper=true,
            linktocpage,
            pagebackref=true]{hyperref}

\begin{document}
\emergencystretch 3em

\author{Menghua Zhai}
\title{Building a Probabilistic Model with Deep Neural Networks for Camera Geo-Calibration}

\abstract
{ \SingleSpacing
Camera calibration is one of the primary task in computer vision.
It plays an important role in applications like SLAM, image
mensuration and indoor layout estimation. Current models for camera
calibration are mostly deterministic, namely they output fixed
solutions. 
However, deterministic calibration models can not handle uncertainties
in our real life well. Furthermore, they are not very friendly to
graphic model algorithms with which they interface.
To address these problems, we propose a probabilistic model with deep
neural networks (DNNs) for camera geo-calibration, providing more
flexible inputs. Compared to traditional camera calibration,
camera geo-calibration measures the camera extrinsic in geographic
coordinates.  Furthermore, our approaches also demonstrate that the
geographic information is helpful for extracting/transferring high
level features in DNN training.  The primary contributions of our
thesis include building a probabilistic model for camera
geo-calibration ,and relating the camera geographic information and
image semantics through deep neural networks. Our discoveries can
provide researchers new insights for both camera calibration and DNN
training.
}
\keywords{geo-calibration, deep neural networks, computer vision}

\advisor{Nathan Jacobs}
\dgs{Miroslaw Truszczynski}

\frontmatter
\maketitle

%\begin{acknowledgments}
	%Acknowledge people/things here
%\end{acknowledgments}

%\begin{dedication}
%	Dedicated to things (optional)
%\end{dedication}

\tableofcontents\clearpage
%\listoffigures\clearpage
%\listoftables\clearpage

\mainmatter

\inject{introduction}
%\inject{background}

%\inject{mcmc}
%\inject{fasthorizon}
%\inject{crosstransf}
%\inject{whenwhere}

%\inject{discussion}

\backmatter

\bibliographystyle{plain}
\bibliography{biblio}

%\inject{vita}

\end{document}
