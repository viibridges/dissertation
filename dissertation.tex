\documentclass[final]{ukthesis}

\usepackage{times}
\usepackage{epsfig}
\usepackage{graphicx}
\usepackage{paralist}
\usepackage{amsmath}
\usepackage{amssymb}
\usepackage{caption}
\usepackage{subcaption}
\usepackage{import}
\usepackage{xspace}
\usepackage{titlesec}
\usepackage{enumitem}
\usepackage{lipsum}
\usepackage{memhfixc}
\usepackage{algorithm2e, algorithmic}
\usepackage{soul}
\usepackage{multirow}
\usepackage{miscstyle}


\usepackage[pdfauthor={Menghua Zhai},
            pdftitle={dissertation},
            bookmarks=false,
            bookmarksnumbered=true,
            bookmarksopen=true,
            breaklinks=true,
            letterpaper=true,
            linktocpage,
            pagebackref=true]{hyperref}

\begin{document}
\emergencystretch 3em

\author{Menghua Zhai}
%\title{Building a Probabilistic Model with Deep Neural Networks for Camera Geo-Calibration}
%\title{Statistical Models for Camera Geo-Calibration}
\title{Deep Probabilistic Models for Camera Geo-Calibration}

\abstract
{ \SingleSpacing
The ultimate goal of image understanding is to transfer the visual
images into numerical or symbolic descriptions of the world
that are helpful for other thought processes for decision making.
In general, understanding images is not trivial for machines.
Under different scenarios, researchers divide image
understanding into different computer vision tasks. For example,
pedestrian detection is essential to autonomous driving, and image
retrieval to an image searching engine. Knowing when, where, and in
which direction picture was taken, the task of geo-calibration makes
it possible to use imagery to understand the world and how it changes
in time. Current models for geo-calibration are mostly deterministic,
namely they output fixed solutions. However, deterministic 
models can not handle uncertainties in our real life well.
Furthermore, their outputs are not very friendly to graphic model
algorithms. To address these problems, we propose a probabilistic
model for camera geo-calibration using deep neural networks (DNNs),
providing more flexible inputs. We also demonstrate that the
geographic information is helpful for extracting/transferring high
level features. The primary contributions of our thesis include
building a probabilistic model for camera geo-calibration, and
relating the camera geographic information and image semantics through
deep neural networks. Our discoveries can provide researchers new
insights for both camera geo-calibration and DNN training.
}
\keywords{geo-calibration, deep neural networks, computer vision}

\advisor{Nathan Jacobs}
\dgs{Miroslaw Truszczynski}

\frontmatter
\maketitle

%\begin{acknowledgments}
	%Acknowledge people/things here
%\end{acknowledgments}

%\begin{dedication}
%	Dedicated to things (optional)
%\end{dedication}

\tableofcontents\clearpage
%\listoffigures\clearpage
%\listoftables\clearpage

\mainmatter

\inject{introduction}
%\inject{background}

%\inject{mcmc}
%\inject{fasthorizon}
%\inject{crosstransf}
%\inject{whenwhere}

%\inject{discussion}

\backmatter

\bibliographystyle{plain}
\bibliography{biblio}

%\inject{vita}

\end{document}
