\documentclass[final]{ukthesis}

\usepackage{times}
\usepackage{epsfig}
\usepackage{graphicx}
\usepackage{paralist}
\usepackage{amsmath}
\usepackage{amssymb}
\usepackage{caption}
\usepackage{subcaption}
\usepackage{import}
\usepackage{xspace}
\usepackage{titlesec}
\usepackage{enumitem}
\usepackage{lipsum}
\usepackage{memhfixc}
\usepackage{algorithm2e, algorithmic}
\usepackage{soul}
\usepackage{multirow}
\usepackage{miscstyle}


\usepackage[pdfauthor={Menghua Zhai},
            pdftitle={dissertation},
            bookmarks=false,
            bookmarksnumbered=true,
            bookmarksopen=true,
            breaklinks=true,
            letterpaper=true,
            linktocpage,
            pagebackref=true]{hyperref}

\begin{document}
\emergencystretch 3em

\author{Menghua Zhai}
\title{Lavaging Geographic Information in Deep Neural Networks for Image Understanding}

\abstract
{ \SingleSpacing
In recent years, Deep Neural Networks (DNNs) have proven to be very
successful in many computer vision areas. Among the most popular
training techniques, supervised learning is wildly used for DNN
training and easy to implement. However, supervised
learning approaches are sensitive to the quality of data
annotations. In some applications like overhead image segmentation and
transient attributes estimation, data with noise-free annotation is
either limited in quantity or technically difficult to acquire.
Our work focuses on lavaging the camera geographic information in
DNN training to deal with the challenges caused by the lack of
noise-free annotations.
In this thesis, the contributions of the geographic information for
DNN training lies in two folds: 1) The geo-calibrated imagery makes it
possible to relate pairs of images from different image sources, and
2) the geographic locations also indicate the appearances of the
scene.
By exploiting the relation between the geographic information and the
scene appearances we are able to develop better approaches to train
deep neural networks than full supervision when quality annotated
data is limited.
Our approach consists of two parts: 1) we explore new methods
to estimate the camera geographic information from images, and 2) 
we propose novel algorithms that lavages the geographic information to
train deep neural networks.
The primary contribution of our thesis is to relate the camera
geographic information and image semantics through deep neural
networks. Our discoveries can provide researchers new insights for
both camera calibration and DNN training.
}
\keywords{geographic information, deep neural networks, computer vision}

\advisor{Nathan Jacobs}
\dgs{Miroslaw Truszczynski}

\frontmatter
\maketitle

%\begin{acknowledgments}
	%Acknowledge people/things here
%\end{acknowledgments}

%\begin{dedication}
%	Dedicated to things (optional)
%\end{dedication}

\tableofcontents\clearpage
%\listoffigures\clearpage
%\listoftables\clearpage

\mainmatter

\inject{introduction}
%\inject{background}

%\inject{mcmc}
%\inject{fasthorizon}
%\inject{crosstransf}
%\inject{whenwhere}

%\inject{discussion}

\backmatter

\bibliographystyle{plain}
\bibliography{biblio}

%\inject{vita}

\end{document}
