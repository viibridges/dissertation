\documentclass[final]{ukthesis}

\usepackage{times}
\usepackage{epsfig}
\usepackage{graphicx}
\usepackage{amsmath}
\usepackage{amssymb}
\usepackage{caption}
\usepackage{subcaption}
\usepackage{import}
\usepackage{xspace}
\usepackage{titlesec}
\usepackage{enumitem}
\usepackage{lipsum}
\usepackage{color}

\usepackage[resetlabels]{multibib}
\newcites{journal,conference,workshop,abstracts}{Refereed
  Journal Publications,Refereed Conference Publications,
  Refereed Workshop Publications,Abstracts}

\usepackage[pdfauthor={Scott Workman},
            pdftitle={Leveraging Overhead Imagery for Localization, Mapping, and Understanding},
            bookmarks=false,
            bookmarksnumbered=true,
            bookmarksopen=true,
            breaklinks=true,
            letterpaper=true,
            linktocpage,
            pagebackref=true]{hyperref}
\usepackage{memhfixc}

\OnehalfSpacing
\setsecnumdepth{subsection}

\DeclareGraphicsExtensions{.pdf,.jpg,.png}
\graphicspath{{images/}}

\newcommand{\inject}[1]{\includefrom{chapters/#1/}{#1}}
\newcommand{\todo}[1]{\textcolor{red}{todo: {\em #1}} }
\newcommand{\chapref}[1]{Chapter~\ref{chap:#1}}
\newcommand{\secref}[1]{Section~\ref{sec:#1}}
\newcommand{\figref}[1]{Figure~\ref{fig:#1}}
\newcommand{\tblref}[1]{Table~\ref{tbl:#1}}
\newcommand{\trans}{\mathsf{T}}
\newcommand{\inv}[1]{\ensuremath{{#1}^{\mathsf{-1}}}}
\newcommand{\Tr}{\operatorname{Tr}}
\newcommand*{\eg}{e.g.\@\xspace}
\newcommand*{\ie}{i.e.\@\xspace}
\newcommand{\ditem}[1]{\item {\sc #1}\hspace{.1in}}

\begin{document}
\emergencystretch 3em

\author{Menghua Zhai}
\title{Geometric Information for Image Understanding}

\abstract
{
  \SingleSpacing
    Deep neural networks as one of the machine learning categories have been proposed
    since 1980s and proved successful in recent years. This thesis proposes methods which
    leverage deep neural networks to solve the camera geo-calibration problem, which is
    one of the import computer vision tasks. Furthermore, we also use the geo-calibration
    information (geo-location, orientation, etc.) to boost the training of the deep neural
    networks for image understanding.
    Our work focuses on three primary research areas: 1) estimating vanishing points
    and horizon line using image global context computed with deep neural networks. 2)
    domain adaptation from the ground image semantics to the satellite image semantics,
    and 3) learning transient feature representations for webcam images using the time
    and geo-location information in a semi-supervised manner.
    The primary contribution of this thesis is to explore the relation between the
    camera calibration information and image semantics, and bridge the gap between
    two research domains using deep neural networks, which provides us new ways for
    camera calibration and network training.
}
\keywords{ deep neural networks, computer vision, geo-calibration}

\advisor{Nathan Jacobs}
\dgs{Miroslaw Truszczynski}

\frontmatter
\maketitle

%\begin{acknowledgments}
	%Acknowledge people/things here
%\end{acknowledgments}

%\begin{dedication}
%	Dedicated to things (optional)
%\end{dedication}

\tableofcontents\clearpage
%\listoffigures\clearpage
%\listoftables\clearpage

\mainmatter

\inject{introduction}
%\inject{background}
%\inject{dependence}
%\inject{discussion}

\backmatter

\bibliographystyle{plain}
\bibliography{biblio}

%\inject{vita}

\end{document}
