\documentclass[final]{ukthesis}

\usepackage{times}
\usepackage{epsfig}
\usepackage{graphicx}
\usepackage{amsmath}
\usepackage{amssymb}
\usepackage{caption}
\usepackage{subcaption}
\usepackage{import}
\usepackage{xspace}
\usepackage{titlesec}
\usepackage{enumitem}
\usepackage{lipsum}
\usepackage{memhfixc}
\usepackage{algorithm2e, algorithmic}
\usepackage{soul}
\usepackage{miscstyle}

\newcommand{\disstitle}{Camera Geometry Detection and Use for Image
Understanding}

\usepackage[pdfauthor={Menghua Zhai},
            pdftitle={\disstitle},
            bookmarks=false,
            bookmarksnumbered=true,
            bookmarksopen=true,
            breaklinks=true,
            letterpaper=true,
            linktocpage,
            pagebackref=true]{hyperref}

\begin{document}
\emergencystretch 3em

\author{Menghua Zhai}
\title{\disstitle}

\abstract
{
  \SingleSpacing
  Camera geometry is one of the fundamental elements in computer vision.
  It plays an essential role in many applications like depth estimation
  and image mensuration. 
  %
  In recent years, the rise of deep neural networks (DNNs) had changed
  the game in many computer vision areas, their ability of abstracting
  high-level semantic knowledges makes a lot of hard challenges
  solvable. 
  %
  This thesis mainly explores the possibility of combining camera
  geometry and DNNs. That is 1) Detecting camera geometry using DNNs,
  and 2) Training DNNs with camera geometric information.  Our work
  includes two research targets: The first part of this thesis focuses
  on camera geometry detection with DNNs, and the rest of it
  demonstrates how camera geometry can boost the learning of DNNs.
  %
  The primary contribution of this thesis is to explore the relation
  between camera geometry and image semantics, and bridge the gap
  between these two research domains, which provides us new ways for
  both camera calibration and deep neural network training.
}
\keywords{ camera geometry, deep neural networks, computer vision}

\advisor{Nathan Jacobs}
\dgs{Miroslaw Truszczynski}

\frontmatter
\maketitle

%\begin{acknowledgments}
	%Acknowledge people/things here
%\end{acknowledgments}

%\begin{dedication}
%	Dedicated to things (optional)
%\end{dedication}

\tableofcontents\clearpage
%\listoffigures\clearpage
%\listoftables\clearpage

\mainmatter

\inject{introduction}
%\inject{mcmc}
%\inject{background}
%\inject{discussion}

\backmatter

\bibliographystyle{plain}
\bibliography{biblio}

%\inject{vita}

\end{document}
